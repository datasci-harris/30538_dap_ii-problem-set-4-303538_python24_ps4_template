% Options for packages loaded elsewhere
\PassOptionsToPackage{unicode}{hyperref}
\PassOptionsToPackage{hyphens}{url}
\PassOptionsToPackage{dvipsnames,svgnames,x11names}{xcolor}
%
\documentclass[
  letterpaper,
  DIV=11,
  numbers=noendperiod]{scrartcl}

\usepackage{amsmath,amssymb}
\usepackage{iftex}
\ifPDFTeX
  \usepackage[T1]{fontenc}
  \usepackage[utf8]{inputenc}
  \usepackage{textcomp} % provide euro and other symbols
\else % if luatex or xetex
  \usepackage{unicode-math}
  \defaultfontfeatures{Scale=MatchLowercase}
  \defaultfontfeatures[\rmfamily]{Ligatures=TeX,Scale=1}
\fi
\usepackage{lmodern}
\ifPDFTeX\else  
    % xetex/luatex font selection
\fi
% Use upquote if available, for straight quotes in verbatim environments
\IfFileExists{upquote.sty}{\usepackage{upquote}}{}
\IfFileExists{microtype.sty}{% use microtype if available
  \usepackage[]{microtype}
  \UseMicrotypeSet[protrusion]{basicmath} % disable protrusion for tt fonts
}{}
\makeatletter
\@ifundefined{KOMAClassName}{% if non-KOMA class
  \IfFileExists{parskip.sty}{%
    \usepackage{parskip}
  }{% else
    \setlength{\parindent}{0pt}
    \setlength{\parskip}{6pt plus 2pt minus 1pt}}
}{% if KOMA class
  \KOMAoptions{parskip=half}}
\makeatother
\usepackage{xcolor}
\setlength{\emergencystretch}{3em} % prevent overfull lines
\setcounter{secnumdepth}{-\maxdimen} % remove section numbering
% Make \paragraph and \subparagraph free-standing
\makeatletter
\ifx\paragraph\undefined\else
  \let\oldparagraph\paragraph
  \renewcommand{\paragraph}{
    \@ifstar
      \xxxParagraphStar
      \xxxParagraphNoStar
  }
  \newcommand{\xxxParagraphStar}[1]{\oldparagraph*{#1}\mbox{}}
  \newcommand{\xxxParagraphNoStar}[1]{\oldparagraph{#1}\mbox{}}
\fi
\ifx\subparagraph\undefined\else
  \let\oldsubparagraph\subparagraph
  \renewcommand{\subparagraph}{
    \@ifstar
      \xxxSubParagraphStar
      \xxxSubParagraphNoStar
  }
  \newcommand{\xxxSubParagraphStar}[1]{\oldsubparagraph*{#1}\mbox{}}
  \newcommand{\xxxSubParagraphNoStar}[1]{\oldsubparagraph{#1}\mbox{}}
\fi
\makeatother

\usepackage{color}
\usepackage{fancyvrb}
\newcommand{\VerbBar}{|}
\newcommand{\VERB}{\Verb[commandchars=\\\{\}]}
\DefineVerbatimEnvironment{Highlighting}{Verbatim}{commandchars=\\\{\}}
% Add ',fontsize=\small' for more characters per line
\usepackage{framed}
\definecolor{shadecolor}{RGB}{241,243,245}
\newenvironment{Shaded}{\begin{snugshade}}{\end{snugshade}}
\newcommand{\AlertTok}[1]{\textcolor[rgb]{0.68,0.00,0.00}{#1}}
\newcommand{\AnnotationTok}[1]{\textcolor[rgb]{0.37,0.37,0.37}{#1}}
\newcommand{\AttributeTok}[1]{\textcolor[rgb]{0.40,0.45,0.13}{#1}}
\newcommand{\BaseNTok}[1]{\textcolor[rgb]{0.68,0.00,0.00}{#1}}
\newcommand{\BuiltInTok}[1]{\textcolor[rgb]{0.00,0.23,0.31}{#1}}
\newcommand{\CharTok}[1]{\textcolor[rgb]{0.13,0.47,0.30}{#1}}
\newcommand{\CommentTok}[1]{\textcolor[rgb]{0.37,0.37,0.37}{#1}}
\newcommand{\CommentVarTok}[1]{\textcolor[rgb]{0.37,0.37,0.37}{\textit{#1}}}
\newcommand{\ConstantTok}[1]{\textcolor[rgb]{0.56,0.35,0.01}{#1}}
\newcommand{\ControlFlowTok}[1]{\textcolor[rgb]{0.00,0.23,0.31}{\textbf{#1}}}
\newcommand{\DataTypeTok}[1]{\textcolor[rgb]{0.68,0.00,0.00}{#1}}
\newcommand{\DecValTok}[1]{\textcolor[rgb]{0.68,0.00,0.00}{#1}}
\newcommand{\DocumentationTok}[1]{\textcolor[rgb]{0.37,0.37,0.37}{\textit{#1}}}
\newcommand{\ErrorTok}[1]{\textcolor[rgb]{0.68,0.00,0.00}{#1}}
\newcommand{\ExtensionTok}[1]{\textcolor[rgb]{0.00,0.23,0.31}{#1}}
\newcommand{\FloatTok}[1]{\textcolor[rgb]{0.68,0.00,0.00}{#1}}
\newcommand{\FunctionTok}[1]{\textcolor[rgb]{0.28,0.35,0.67}{#1}}
\newcommand{\ImportTok}[1]{\textcolor[rgb]{0.00,0.46,0.62}{#1}}
\newcommand{\InformationTok}[1]{\textcolor[rgb]{0.37,0.37,0.37}{#1}}
\newcommand{\KeywordTok}[1]{\textcolor[rgb]{0.00,0.23,0.31}{\textbf{#1}}}
\newcommand{\NormalTok}[1]{\textcolor[rgb]{0.00,0.23,0.31}{#1}}
\newcommand{\OperatorTok}[1]{\textcolor[rgb]{0.37,0.37,0.37}{#1}}
\newcommand{\OtherTok}[1]{\textcolor[rgb]{0.00,0.23,0.31}{#1}}
\newcommand{\PreprocessorTok}[1]{\textcolor[rgb]{0.68,0.00,0.00}{#1}}
\newcommand{\RegionMarkerTok}[1]{\textcolor[rgb]{0.00,0.23,0.31}{#1}}
\newcommand{\SpecialCharTok}[1]{\textcolor[rgb]{0.37,0.37,0.37}{#1}}
\newcommand{\SpecialStringTok}[1]{\textcolor[rgb]{0.13,0.47,0.30}{#1}}
\newcommand{\StringTok}[1]{\textcolor[rgb]{0.13,0.47,0.30}{#1}}
\newcommand{\VariableTok}[1]{\textcolor[rgb]{0.07,0.07,0.07}{#1}}
\newcommand{\VerbatimStringTok}[1]{\textcolor[rgb]{0.13,0.47,0.30}{#1}}
\newcommand{\WarningTok}[1]{\textcolor[rgb]{0.37,0.37,0.37}{\textit{#1}}}

\providecommand{\tightlist}{%
  \setlength{\itemsep}{0pt}\setlength{\parskip}{0pt}}\usepackage{longtable,booktabs,array}
\usepackage{calc} % for calculating minipage widths
% Correct order of tables after \paragraph or \subparagraph
\usepackage{etoolbox}
\makeatletter
\patchcmd\longtable{\par}{\if@noskipsec\mbox{}\fi\par}{}{}
\makeatother
% Allow footnotes in longtable head/foot
\IfFileExists{footnotehyper.sty}{\usepackage{footnotehyper}}{\usepackage{footnote}}
\makesavenoteenv{longtable}
\usepackage{graphicx}
\makeatletter
\def\maxwidth{\ifdim\Gin@nat@width>\linewidth\linewidth\else\Gin@nat@width\fi}
\def\maxheight{\ifdim\Gin@nat@height>\textheight\textheight\else\Gin@nat@height\fi}
\makeatother
% Scale images if necessary, so that they will not overflow the page
% margins by default, and it is still possible to overwrite the defaults
% using explicit options in \includegraphics[width, height, ...]{}
\setkeys{Gin}{width=\maxwidth,height=\maxheight,keepaspectratio}
% Set default figure placement to htbp
\makeatletter
\def\fps@figure{htbp}
\makeatother

\usepackage{fvextra}
\DefineVerbatimEnvironment{Highlighting}{Verbatim}{breaklines,commandchars=\\\{\}}
\KOMAoption{captions}{tableheading}
\makeatletter
\@ifpackageloaded{caption}{}{\usepackage{caption}}
\AtBeginDocument{%
\ifdefined\contentsname
  \renewcommand*\contentsname{Table of contents}
\else
  \newcommand\contentsname{Table of contents}
\fi
\ifdefined\listfigurename
  \renewcommand*\listfigurename{List of Figures}
\else
  \newcommand\listfigurename{List of Figures}
\fi
\ifdefined\listtablename
  \renewcommand*\listtablename{List of Tables}
\else
  \newcommand\listtablename{List of Tables}
\fi
\ifdefined\figurename
  \renewcommand*\figurename{Figure}
\else
  \newcommand\figurename{Figure}
\fi
\ifdefined\tablename
  \renewcommand*\tablename{Table}
\else
  \newcommand\tablename{Table}
\fi
}
\@ifpackageloaded{float}{}{\usepackage{float}}
\floatstyle{ruled}
\@ifundefined{c@chapter}{\newfloat{codelisting}{h}{lop}}{\newfloat{codelisting}{h}{lop}[chapter]}
\floatname{codelisting}{Listing}
\newcommand*\listoflistings{\listof{codelisting}{List of Listings}}
\makeatother
\makeatletter
\makeatother
\makeatletter
\@ifpackageloaded{caption}{}{\usepackage{caption}}
\@ifpackageloaded{subcaption}{}{\usepackage{subcaption}}
\makeatother

\ifLuaTeX
  \usepackage{selnolig}  % disable illegal ligatures
\fi
\usepackage{bookmark}

\IfFileExists{xurl.sty}{\usepackage{xurl}}{} % add URL line breaks if available
\urlstyle{same} % disable monospaced font for URLs
\hypersetup{
  pdftitle={PS4 Andy Fan Will Sigal},
  colorlinks=true,
  linkcolor={blue},
  filecolor={Maroon},
  citecolor={Blue},
  urlcolor={Blue},
  pdfcreator={LaTeX via pandoc}}


\title{PS4 Andy Fan Will Sigal}
\author{}
\date{}

\begin{document}
\maketitle

\RecustomVerbatimEnvironment{verbatim}{Verbatim}{
  showspaces = false,
  showtabs = false,
  breaksymbolleft={},
  breaklines
}


\textbf{PS4:} Due Sat Nov 2 at 5:00PM Central. Worth 100 points.

\subsection{Style Points (10 pts)}\label{style-points-10-pts}

\subsection{Submission Steps (10 pts)}\label{submission-steps-10-pts}

\subsection{Download and explore the Provider of Services (POS) file (10
pts)}\label{download-and-explore-the-provider-of-services-pos-file-10-pts}

(This file records all providers certified to bill Medicare and
Medicaid. Each provider has a unique CMS certification number,
consistent across different years)

(The data dictionary is located at the bottom of the page as ``Provider
of Services File- Hospital \& Non-Hospital Facilities Data Dictionary)

\begin{Shaded}
\begin{Highlighting}[]
\CommentTok{\#\#\# SETUP }
\ImportTok{import}\NormalTok{ pandas }\ImportTok{as}\NormalTok{ pd}
\ImportTok{import}\NormalTok{ altair }\ImportTok{as}\NormalTok{ alt}
\ImportTok{import}\NormalTok{ time}
\ImportTok{import}\NormalTok{ os}
\ImportTok{import}\NormalTok{ warnings }
\NormalTok{warnings.filterwarnings(}\StringTok{\textquotesingle{}ignore\textquotesingle{}}\NormalTok{)}
\end{Highlighting}
\end{Shaded}

\paragraph{1. (Partner 1) This is a fairly large dataset and we won't be
using most of the variables. Read through the rest of the problem set
and look through the data dictionary to identify which variables you
will need to complete the exercise, and use the tool on data.cms.gov
into restrict to those variables (``Manage Columns'') before exporting
(``Export''). Download this for 2016 and call it pos2016.csv. What are
the variables you
pulled?}\label{partner-1-this-is-a-fairly-large-dataset-and-we-wont-be-using-most-of-the-variables.-read-through-the-rest-of-the-problem-set-and-look-through-the-data-dictionary-to-identify-which-variables-you-will-need-to-complete-the-exercise-and-use-the-tool-on-data.cms.gov-into-restrict-to-those-variables-manage-columns-before-exporting-export.-download-this-for-2016-and-call-it-pos2016.csv.-what-are-the-variables-you-pulled}

required variables:

provider type code: PRVDR\_CTGRY\_CD

subtype code: PRVDR\_CTGRY\_SBTYP\_CD

CMS certification number: PRVDR\_NUM

Termination code: PGM\_TRMNTN\_CD

zip code: ZIP\_CD

\begin{Shaded}
\begin{Highlighting}[]
\CommentTok{\#\#\# import data}
\NormalTok{path1 }\OperatorTok{=} \StringTok{\textquotesingle{}d:}\CharTok{\textbackslash{}\textbackslash{}}\StringTok{UChicago}\CharTok{\textbackslash{}\textbackslash{}}\StringTok{Classes}\CharTok{\textbackslash{}\textbackslash{}}\StringTok{2024Qfall}\CharTok{\textbackslash{}\textbackslash{}}\StringTok{Programming Python}\CharTok{\textbackslash{}\textbackslash{}}\StringTok{problem{-}set{-}4{-}andy{-}fan{-}will{-}sigal}\CharTok{\textbackslash{}\textbackslash{}}\StringTok{pos2016.csv\textquotesingle{}}
\NormalTok{df }\OperatorTok{=}\NormalTok{ pd.read\_csv(path1, encoding}\OperatorTok{=}\StringTok{"utf{-}8"}\NormalTok{, on\_bad\_lines}\OperatorTok{=}\StringTok{"skip"}\NormalTok{)}
\end{Highlighting}
\end{Shaded}

\paragraph{2. (Partner 1) Import your pos2016.csv file. We want to focus
on short-term hospitals. These are identified as facilities with
provider type code 01 and subtype code 01. Subset your data to these
facilities. How many hospitals are reported in this data? Does this
number make sense? Cross-reference with other sources and cite the
number you compared it to. If it differs, why do you think it could
differ?}\label{partner-1-import-your-pos2016.csv-file.-we-want-to-focus-on-short-term-hospitals.-these-are-identified-as-facilities-with-provider-type-code-01-and-subtype-code-01.-subset-your-data-to-these-facilities.-how-many-hospitals-are-reported-in-this-data-does-this-number-make-sense-cross-reference-with-other-sources-and-cite-the-number-you-compared-it-to.-if-it-differs-why-do-you-think-it-could-differ}

\begin{verbatim}
a.how many hospitals, does it make sense

b.cross ref. other sources, if it differs, why?
\end{verbatim}

\paragraph{3. (Partner 1) Repeat the previous 3 steps with 2017Q4,
2018Q4, and 2019Q4 and then append them together. Plot the number of
observations in your dataset by
year.}\label{partner-1-repeat-the-previous-3-steps-with-2017q4-2018q4-and-2019q4-and-then-append-them-together.-plot-the-number-of-observations-in-your-dataset-by-year.}

\paragraph{4. (Partner 1) Each hospital is identified by its CMS
certification number. Plot the number of unique hospitals in your
dataset per year. Compare this to your plot in the previous step. What
does this tell you about the structure of the
data?}\label{partner-1-each-hospital-is-identified-by-its-cms-certification-number.-plot-the-number-of-unique-hospitals-in-your-dataset-per-year.-compare-this-to-your-plot-in-the-previous-step.-what-does-this-tell-you-about-the-structure-of-the-data}

\begin{verbatim}
a.plot unique hospitals

b.compare to previous step, what does this tell about data structure
\end{verbatim}

\subsection{Identify hospital closures in POS file (15 pts)
(*)}\label{identify-hospital-closures-in-pos-file-15-pts}

\begin{enumerate}
\def\labelenumi{\arabic{enumi}.}
\tightlist
\item
\item
\item
  \begin{enumerate}
  \def\labelenumii{\alph{enumii}.}
  \tightlist
  \item
  \item
  \item
  \end{enumerate}
\end{enumerate}

\subsection{Download Census zip code shapefile (10
pt)}\label{download-census-zip-code-shapefile-10-pt}

\begin{enumerate}
\def\labelenumi{\arabic{enumi}.}
\tightlist
\item
  \begin{enumerate}
  \def\labelenumii{\alph{enumii}.}
  \tightlist
  \item
  \item
  \end{enumerate}
\item
\end{enumerate}

\subsection{Calculate zip code's distance to the nearest hospital (20
pts)
(*)}\label{calculate-zip-codes-distance-to-the-nearest-hospital-20-pts}

\begin{enumerate}
\def\labelenumi{\arabic{enumi}.}
\tightlist
\item
\item
\item
\item
  \begin{enumerate}
  \def\labelenumii{\alph{enumii}.}
  \tightlist
  \item
  \item
  \item
  \end{enumerate}
\item
  \begin{enumerate}
  \def\labelenumii{\alph{enumii}.}
  \tightlist
  \item
  \item
  \item
  \end{enumerate}
\end{enumerate}

\subsection{Effects of closures on access in Texas (15
pts)}\label{effects-of-closures-on-access-in-texas-15-pts}

\begin{enumerate}
\def\labelenumi{\arabic{enumi}.}
\tightlist
\item
\item
\item
\item
\end{enumerate}

\subsection{Reflecting on the exercise (10
pts)}\label{reflecting-on-the-exercise-10-pts}




\end{document}
